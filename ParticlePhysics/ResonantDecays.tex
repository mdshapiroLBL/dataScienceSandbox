\documentclass[12pt]{article}
\usepackage{graphicx}
\addtolength{\evensidemargin}{-0.5in}
\addtolength{\oddsidemargin}{-0.5in}
\begin{document}
\begin{center}
{\large\bf Resonance Decays and Dalitz Plots}
\end{center}
\vskip0.2in
% Taken from Yury Kolmensky's Physics 129 Class
Goals:
\begin{itemize}
\item Gain experience in using the Particle Data Group compilations of decay rates to determine
  relative branching fractions for different decay modes and to develop insight on decay kinematics.
\item Using the example of $D_s^{+}\rightarrow K^+K^-\pi^{+}$, review how Dalitz Plots can be used to understand resonant structure in hadron decays.
\end{itemize}
B-factories are a good place to study the decays of bottom hadrons.  These $e^+e^-$
colliders typically
run at a center-of-mass energy of 10.579~GeV, the mass of the $\Upsilon (4s)$.  This
$b\overline b $ resonance is just above open bottom threshold
and therefore decays (more than 96\%\ of the time) to
a $B\overline B$ (there is not enough additional energy to produce even a single additional pion).
Consider the case $\Upsilon (4s)\rightarrow B^0\overline B^0$.  If the $\overline B^0$ is fully reconstructed,
then we know that all the remaining particles must come from the decay of the $B^0$.  This sample
can therefore be used to measure absolute branching fractions.
\begin{enumerate}
\item Consider the production $e^+e^- \rightarrow  B\overline B$ followed by the
decay chain:
\begin{eqnarray*}
B^0 & \rightarrow & D_s^{*+}\pi^-\\
D_s^{*+} & \rightarrow & D_s^+ \gamma\\
D_s  & \rightarrow & K^+K^- \pi^+
\end{eqnarray*}
\begin{enumerate}
\item For each of the particles listed above, use the Particle Data Group (PDG) tables to find their mass and spin. 
\item Using the averages from the PDG, compare the branching fractions for  $D_s^{*+}\rightarrow D_s^+ \gamma$
and $D^{*+}\rightarrow D_s^+ \gamma$.  Why are these so different?
%\item Assuming that the $B^0$ is produced at rest in the lab frame, what is the momentum of the $D_S^{*+}$? in
%the lab?
\end{enumerate}
\item The $D_s\rightarrow K^+K^-\pi$ decay can occur through several possible intermediate resonance states.
In principle theses should be combined by adding their amplitudes since they can quantum mechanically
interfer.  However, the resonant structure is still evident in the decay kinematics and it is possible
to observe this structure using a Dalitz plot. The file  {\it DsstarPiMC.root} contains a simulated
sample that corrresponds to the decay chain above.
\begin{enumerate}
\item Make the Dalitz plot for variables $m_{KK}^2$ and $m_{\pi K}^2$
(combining the kaon and the pion of the opposite charge). Explain the structure.
\item Compute the average invariant mass of the resonant structure in the $K^+K^-$ final state,
and compare it to the mass of the $\phi$ meson in PDG. 
\item Compute the average invariant mass of the resonant structure in the $K^-\pi^+$ final state,
and compare it to the mass of the $K^*$ meson in PDG.  Check if the invariant mass you get is the same if you select a $K^+\pi^-$ final state.
\item Using these data, compute the branching fractions for decays
\begin{eqnarray*}
D_s^+ &\rightarrow &  \phi \pi^+\\ 
D_s^+ & \rightarrow &  \overline K^{*0}  K^+  
\end{eqnarray*}
as well as the nonresonant
$$
D_s+ \rightarrow K^+ K^- \pi^+
$$
and compare them to the values given in the PDG.
\end{enumerate}
\item Assuming that the $B^0$ is produced at rest in the lab frame, what is the momentum of the $D_S^{*+}$? in
the lab frame?  Check this prediction using the simulated sample?
\item Consider the decay $D_s^{*+} \rightarrow  D_s^+ \gamma$.  Do you expect the angular distribution of
  this decay to be isotropic in the $D_s^{*+}$ rest frame?  You do not need to calculate the form of
  the distribution but rather give a physical argument that supports your answer.
\item Check using the simulated answer whether your prediction is correct.
\end{enumerate}
\end{document}
