\documentclass[12pt]{article}
\usepackage{graphicx}
\begin{document}
\begin{center}
  {\large\bf Parity Violation in Electron Scattering}
\end{center}
\vskip0.2in
% Taken from Yury Kolmensky's Physics 129 Class
Goals:
\begin{itemize}
\item Gain experience in data analysis using a data sample
  from the E158 experiment at SLAC
\item Introduce the concept of a parity violating asymmetry
\item Explore strategies for correcting measurements for systematic
  variations.
  Use the measured correlations to translate between a "raw" (aka "measured") asymmetry and a "corrected"
  (aka "true") asymmetry
\item Learn how to determine confidence intervals for measured quantities
\end{itemize}
The E158  experiment at SLAC measured a parity-violating asymmetry in
M\o ller (electron-electron) scattering. This was a fixed-target experiment,
which scattered longitudinally-polarized electrons off atomic (unpolarized)
electrons in a 1.5m liquid hydrogen target. The files provided
here contain a
snapshot of 10,000 "events" from this experiment (overall, the experiment
collected almost 400 million such events over the course of about 4 months).
Each event actually records a pair of pulses: one for the right-handed electron
and one for the left-handed electron. For each event, 4 variables are recorded:
\begin{itemize}
    \item{\bf Counter:} A unique number labeling the event
    \item{\bf Asym:} The "raw" cross section asymmetry from one of the
      detector channels (there are 50 of these overall). The cross section
      asymmetry is defined as
      $$
      A_{raw} = \frac{\sigma_R -\sigma_L}{\sigma_R +\sigma_L}
      $$
      The asymmtry is recorded in units of PPM (parts per million). It is
      called "raw" because corrections due to the difference in beam properties
      at the target are not yet applied (see below).
    \item{$\bf \Delta X$:}  The beam position at the target in the $x$-direction
      in microns (with the convention that the beam is traveling along Z).
    \item{$\bf \Delta Y$:} The beam position at the target in the $y$-direction in microns 
\end{itemize}
The data sample is provided in internal ROOT format (asymdata.root)
and in plain text format (asymdata.txt).
You are free to use either to answer the questions below.
\begin{enumerate}
\item  Compute the mean of the raw asymmetry, Asym,
  distribution and its statistical uncertainty.
\item Compute the correlation coefficients: ${\rm Corr}(Asym,\Delta X)$,
  ${\rm Corr}(Asym,\Delta Y)$, and ${\rm Corr}(\Delta X,\Delta Y)$.
  Which variables are approximately independent of each other?
\item A better estimator of the parity-violating asymmetry is the quantity that
  corrects for possible differences in the right- and left-handed
  electron pulses. We want to define a quantity
  $$  A_{PV} = A_{raw} - a_x \Delta X - a_y \Delta Y $$
  with coefficients $a_x$ and $a_y$ defined in such a way that
  $A_{PV}$ is independent of $\Delta X$ and $\Delta Y$.
  This is called a linear regression. For now, we will assume that $a_x$ is negligible (how good is this assumption?) and compute $a_y$. Compute $a_y$ and its
  uncertainty using:
   \begin{enumerate}
   \item A binned least squares fit, using a profile histogram
   \item An unbinned least squares fit
   \end{enumerate}
 \item Compute the mean of the regressed asymmetry distribution $A_{PV}$ and
   its statistical uncertainty.
\item Compute the 90\%\ confidence interval for the true value of $A_{PV}$
\item Check the hypothesis that the distribution of $A_{PV}$ is Gaussian. For this, you can bin the distribuition of $A_{PV}$ in a reasonable number of bins (e.g. 100) and compute the goodness-of-fit (or p) value for the Gaussian distribution. Estimate the confidence level (or probability that the hypothesis is correct). 
\end{enumerate}
\end{document}
