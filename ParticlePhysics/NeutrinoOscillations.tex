\documentclass[12pt]{article}
\usepackage{graphicx}
\begin{document}
\begin{center}
{\large\bf Neutrino Oscillations}
\end{center}
\vskip0.2in
% Taken from Yury Kolmensky's Physics 129 Class
Goals:
\begin{itemize}
\item Understand the phenomology of neutrino oscillations by
  studying the expression of the mixing probability as a function
  of distance from the neutrino production point.
\item Learn how the best estimate of $\Delta m^2$ and its uncertainty
can be obtained using a fit to experimental data.
\end{itemize}
Consider a system with two massive neutrinos,$\nu_e$ and $\nu_\mu$.
An electron neutrino, $\nu_e$, created at time $t= 0$ in the core
of the Sun, is described as a linear combination of two mass
eigenstates $\nu_1$and $nu_2$:
$$
\left | \nu_e \right \rangle = \cos\theta \left | \nu_1 \right \rangle + \sin \theta  \left |\nu_2 \right \rangle
$$
where $\theta$ is a two-generation mixing angle.
Similarly, the muon neutrino would be defined as the orthogonal
linear combination:
$$
\left | \nu_\mu \right \rangle = -\sin\theta \left | \nu_1 \right \rangle +
\cos \theta  \left |\nu_2 \right \rangle
$$
\begin{enumerate}
\item[(a)] Show that the probability of finding a muon neutrino $\nu_mu$
at a distance $L=t$ from the source is given by
$$
P_{\nu_e\rightarrow \nu_\mu}(t) = \sin^2\left (2\theta\right )
\sin^2\left (\frac{\Delta m^2 L}{4E}\right )
$$
where $E\approx p >>m_1,m_2$ is the neutrino energy and
$\Delta m^2 = {m^2}_1 - {m^2}_2$.
\item[(b)] With $E$ measured in GeV, $L$ in km,and in $\Delta m^2$ eV$^2$,
show that the previous formula can be re-written as
$$
P_{\nu_e\rightarrow \nu_\mu}(t) = \sin^2\left (2\theta \right )
\sin^2 \left (\frac{1.27 \Delta m^2 L}{E} \right )
$$
This formula applies to any two-flavor neutrino mixing.
\item[(c)] Super-Kamiokande (SuperK) experiment has analyzed
the atmospheric ($\nu_e$ and $\nu_mu$) neutrino data and
shown that they are consistent with
two-flavor $\nu_\mu \rightarrow \nu_\tau $ oscillations with
$\sin^2(2\theta)\sim1$ and $\Delta m^2\approx 10^{-3}$.
They have also looked at the ratio of the number of
$\nu_\mu$ events observed in data to the Monte Carlo prediction.
The muon neutrinos were detected through the charge-current reaction
emitting a muon. The approximate results are shown below.
The distance from the production point to the detector was varied by
observing the azimuthal dependence of the neutrino flux. Here,
Monte Carlo  takes into account
muon production in the atmosphere, energy-dependent efficiencies,
etc, but assumes no oscillations.
Thus, if neutrinos oscillate, the ratio of the yield
in data to that in Monte Carlo should vary with distance
according to the formula you derived above.
\begin{center}
  \begin{tabular}{|l|l|}
    \hline
  $L/E_\nu$ (km/GeV) & Data / Monte Carlo\\
 \hline
 0.7& 1.00$\pm$0.20 \\
20 & 1.00$\pm$0.08\\
70 & 0.90$\pm$0.08\\
200 & 0.83$\pm$0.15\\
700& 0.65$\pm$0.15\\
2000& 0.70$\pm$0.08\\
7000& 0.63$\pm$0.06\\
20000& 0.61$\pm$0.08\\
\hline
\end{tabular}
\end{center}
Make a plot of the results above with $L/E_\nu$ on the horizontal
axis in log scale. At large $L/E_\nu$, the muon neutrinos have
presumably undergone numerous oscillationsand have averaged out to
roughly half the initial rate.  Using
$\sin^2\left (2\theta \right ) = 1$, make an estimate of $\Delta m^2$.
\end{enumerate}
\end{document}
