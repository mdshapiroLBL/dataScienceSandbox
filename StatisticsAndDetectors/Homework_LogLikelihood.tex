\documentclass[12pt]{article}
\usepackage{graphicx}
\renewcommand{\theenumi}{\alph{enumi}}
\begin{document}
\begin{center}
{\large\bf Log Likelihood Fitting
}
\end{center}
\vskip0.2in
Consider the problem of determining the lifetime of a species 
of particle that we can stop in our detector
by observing it decays in the presence of a constant background. The rate is
given by
$$
R(t) = A + Be^{-\Gamma t}
$$
We'll take as the true parameters $A=1/$sec, $B=2/$sec, and $\Gamma=2/$sec.
Each time we stop a particle in our detector, we will only wait to see
if it decays for a maximum time $t_{max}=3$ seconds.
We can imagine doing the experiment over many times (each experiment takes 
three seconds) to accumulate a lot of data. 
\begin{enumerate}
\item First, let's generate some fake data. 
We will generate this data using the Monte Carlo technique known 
as the ``acceptance-rejection method'' or  ``rejection sampling.''
Suppose you want to generate events whose distribution follows the
function $f(x)$ (here $f(x)$ is the probability distribution function).
This can be achieved by generating
points $(x_i,y_i)$
randomly in a two-dimensional space and keeping only the
subset of the events where $y_i\le f(x_i)$.  A more detailed
explaination of the method can be found here:
\begin{flushleft}
https://en.wikipedia.org/wiki/Rejection\_sampling
\end{flushleft}
We will generate fake data in the following way.  
The maximum rate is at $t=0$, where it is $R_{max}=(A+B)$.  
Pick an $x_i$, the time $t$ of the decay, randomly 
between 0 and $t_{max}$, and pick $y_i$ randomly between 0 and $R_{max}$.
Use $R(t)$  to you decide whether to keep this event.
Generate a reasonable amount of such fake data.
What percentage of the time do you
expect to keep an event? Is that what you find?
\item  Calculate the negative log-likelihood function, $-\ln {\cal L}$,
for your data taking care to use a PDF normalized so
$$
\int_0^{t_{max}}f(t)dt=1
$$
Using this condition you can express your likelihood in terms of two 
free parameters, $\Gamma$ and $\kappa=A/B$.  (Note:  you can
see an example of how this works in the handout {\it Some
Comments on Likelihood Functions} that is posted on our bCourses
page).\\
We will study the simulated data, pretending that we don't know what 
values of $A$, $B$ and $\Gamma$ were used to generate it.  We want
to determine $\Gamma$ from the data.
Write code to calculate the negative log-likelihood:
$$
- \ln {\cal L} = - \sum_i \ln f(\kappa, \Gamma, t_i)
$$
where the $t_i$ are the time values you accepted in the Monte Carlo process.
\item There are lots of algorithms for finding the 
minimum of a non-linear function such as our negative log-likelihood, 
but we won't bother to use any of
these algorithms here.
Instead, we will explore the minimum by inspecting the behavour of
the function.
Plot (or just show the values in a 2-dimensional grid) 
the value of $-\ln {\cal L}$ you obtain from your simulated data
as you vary $\kappa$ and $\Gamma$  
in the region  of the true answer ($\kappa=0.5$, $\Gamma=2$).  
How close is the $\Gamma $ that gives minimum negative
log-likelihood  to the true value of $\Gamma$? 
Now do the same,
but with ten times as much simulated data. How close is the maximum now to
the true value?
\item  We saw in class that for high statistics $-2\ln {\cal L}$ is 
distributed like a $\chi^2$ distribution and the uncertainty on
the estimate of a parameter of the function can be obtained by finding how much
the you can change the parameter to increase $-2\ln {\cal L}$ by
1. 
Assuming you know $\kappa$ exactly, plot  $-2\ln {\cal L}$ for
values of $\Gamma$ around the maximum likelihood and indicate
the uncertainty on the measured value of $\Gamma$.

\noindent
Try running your code with different numbers
of events per pseudoexperiment and running many pseudoexperiments. 
Is it true that 68\%\ of the time
the true value lies within $\Delta {\cal L} = 1$? 
\end{enumerate}
\end{document}
